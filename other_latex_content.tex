\subsection{Evaluation of alert forecasting}
\label{sec:eval-extr-value}

The alert forecasting is a classifier problem where we evaluate if the levels of NO2 are above

\subsubsection{Alert Protocol}

The Madrid region is divided into zones and each zone contains pollution stations that track the levels
of pollutants.

There are 3 levels of alert based on the levels of NO2:
\begin{itemize}
  \item Prewarning (More than 180 $\frac{\mu gr}{m^3}$ ): 2 consecutive hours on 2 stations in the same zone OR 3 
  consecutive hours in 3 stations on any zone.
  \item Warning (More than 200 $\frac{\mu gr}{m^3}$ ): 2 consecutive hours on 2 stations in the same zone OR 3 
  consecutive hours in 3 stations on any zone.
  \item Alert (More than 400 $\frac{\mu gr}{m^3}$ ): 3 consecutive hours on 3 stations in the same zone OR 2 
  consecutive hours in zone 4. 
\end{itemize} 

\subsubsection{Training data}
\label{sec:eval-extr-value}

We have few alerts in the last 5 years, so it is difficult to evaluate meaningfully the alert prediction.

\subsection{Probabilistic forecasting of extreme values}
\label{sec:probabilistic}


\subsection{Forecasting the probability of alerts}
\label{sec:alertProb2}