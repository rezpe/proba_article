\documentclass[a4paper,twocolumn,5p]{elsarticle}

\usepackage{hyperref}
%\usepackage{lineno}
%\modulolinenumbers[5]

\usepackage{booktabs}
\usepackage{graphicx}
\usepackage{xspace}
\usepackage{booktabs}
\usepackage[draft]{fixme}

\journal{Environment International}

%% `Elsevier LaTeX' style
\bibliographystyle{elsarticle-num}
%%%%%%%%%%%%%%%%%%%%%%%

\begin{document}

% Macro para escribir NO$_2$
\newcommand{\no}{NO\textsubscript{2}\xspace}

\begin{frontmatter}

\title{Probabilistic forecasting for extreme \no pollution episodes}

\author{Jos\'e L. Aznarte\fnref{myfootnote}}
\address{Artificial Intelligence Department\\Universidad Nacional de
  Educaci\'on a Distancia --- UNED\\c/ Juan del Rosal, 16, Madrid, Spain}
\ead{jlaznarte@dia.uned.es}
\fntext[myfootnote]{This work has been partially funded by Ministerio
  de Econom\'ia y Competitividad, Gobierno de Espa\~na, through a
  \emph{Ram\'on y Cajal} grant % awarded to Dr Aznarte
  (reference: RYC-2012-11984).}


\begin{abstract}
  In this study, we investigate the feasibility and convenience of
  quantile regression to predict extreme concentrations of
  \no. Contrarily to the usual point-forecasting, where a single value
  (mean or median) is forecast for each horizon, probabilistic
  forecasting through quantile regression allows for the prediction of
  the full probability distribution, which in turn allows to build
  models specifically fit for the tails of this distribution.

  Using data from the atmospheric pollution monitoring system of
  Madrid, including \no concentrations as well as meteorological
  measures, we demonstrate that such models are able to predict
  extreme \no concentrations, outperforming point-forecasting
  alternatives, and that the predictions are accurate, reliable and
  sharp. Besides, we study the relative importance of the independent
  variables involved, and show how the important variables for the
  median quantile are different than those important for the upper
  quantiles. Furthermore, we show a method to compute the probability
  of exceedance of thresholds, which is proven to be a simple and
  comprehensible manner to present probabilistic forecasts maximizing
  their usefulness.

  % In this paper a new approach towards pollution forecasting is
  % presented. Contrarily to the usual point-forecasting, where a single
  % value is forecast for each horizon, in this case the full
  % probability is predicted. This permits focusing on parts of this
  % distribution which are of interest, i.e., the upper quantiles, which
  % contain extreme values that trigger traffic restrictions
\end{abstract}

\begin{keyword}
probabilistic forecasting \sep air quality \sep quantile regression
\sep nitrogen dioxide \sep Madrid
\end{keyword}

\end{frontmatter}

%\linenumbers

\section{Introduction}
\label{sec:intro}

With more and more problematic pollution levels in cities around the
world, and given the scientific consensus about their adverse effects
on health \cite{Kampa2008362,Kim2015136,Sellier2014165}, traffic restrictions
emerge as a temporary remedy when high pollution episodes occur. For
example, Madrid has enforced a new air quality protocol which includes
reducing speed limits and other restrictions to the circulation of
pollutant vehicles when \no concentrations reach certain thresholds
during a certain period. Anticipating the activation of such
restrictions is critical both for the decision makers (which need to
announce them in advance) and for the vehicle owners (which need to
plan their transport alternatives).

Notwithstanding, research on forecasting extreme pollution events is
meagre in general and, in particular, to our knowledge, probabilistic
forecasting has never been put in practice to deal with \no
concentrations. Obviously, forecasting the central tendencies of the
data distribution, i.e. the conditional mean, is not the best approach
if the main interest is to predict possible exceedances of thresholds
that lie on the tails of the distribution.

% Urban air pollution is a major concern for its adverse effects on
% health. The World Health Organization (WHO) estimates that air
% pollution caused 3.7 million premature deaths worldwide in 2012
% \cite{who_ambient_2016} and more and more studies prove its negative
% effects on health \cite{bentayeb_association_2015}. Nitrogen dioxide
% (\no2), together with ozone and particulate matter, is one of the
% main air pollutants: it is a toxic gas which causes significant
% inflammation of the airways, and is the source of nitrate aerosols,
% which form an important fraction of particulate matter and, in the
% presence of ultraviolet light, ozone.

% Epidemiological studies have shown that symptoms of bronchitis in
% children increase in association with exposure to \no2
% \cite{pershagen_air_1995,brunekreef_air_1997}. There have been few
% controlled human exposure studies on interactions with other chemical
% pollutants, although studies show that \no2 exposure enhances
% allergic responses of asthmatics to inhaled pollens
% \cite{strand_nitrogen_1997}.

% Nitrogen dioxide belongs to a family of highly reactive gases called
% nitrogen oxides (NO$_\mathrm{x}$). These gases form when fuel is burned at high
% temperatures, and come principally from motor vehicle exhaust and
% stationary sources such as electric utilities and industrial boilers.
% According to WHO, in most urban environments in Europe the principal
% source of \no2 is NO$_\mathrm{x}$ from motor vehicles of all types.  Hence
% city planners work to improve air quality by reducing traffic
% emmisions, as evidence shows that traffic restrictions are useful in
% reducing \no2 concentrations \cite{wang_traffic_2007}.  This is the
% case in Madrid, where the local government recently imposed new
% anti-pollution regulations which include traffic restrictions when
% pollutants reach the thresholds set by the European Commission
% \cite{european_commission_air_2008}.

% % Driving restrictions have been widely used as a method to reduce
% % urban air pollution and traffic congestion in developing
% % countries. Santiago, Chile introduced a driving restriction in 1986
% % and Mexico City, Mexico introduced a driving restriction in
% % 1989. Following these two, several more Latin American cities have
% % introduced driving restrictions, including Bogot�, Colombia and S�o
% % Paulo, Brazil. Beijing and its neighboring city Tianjin also
% % implemented driving restrictions during the 2008 Olympic Games and a
% % modified version of the restriction continued in Beijing after the
% % Olympics.

% But driving restrictions are a severe measure which impacts the daily
% life of millions of people which need to adapt their usual travel
% patterns. Hence, being able to provide anticipating information about
% future activation of the restrictions is in the interest of citizens,
% which can plan their journeys in advance, and decision makers, as it
% helps avoiding the exceedance of critical thresholds and has an effect
% on popular perception of the measures. For this reason, air quality
% forecasting is an active field of study which grew steadily during the
% last three decades and which gathers contributions from meteorology,
% physics, chemistry, statistics and computational intelligence.

% There are several approaches to air quality forecasting and they can
% be roughly divided in two classes: those which rely upon analyzing the
% atmosphere status and evolution from a fluid mechanics and chemical
% point of view and those which study pollution measures from a
% statistical time series analysis and modelling perspective. The latter
% can be subsequently divided in approaches that assume that the data
% are generated by a given stochastic data model and those that use
% algorithmic models and treat the data mechanisms as unknown
% \cite{breiman_statistical_2001}. These algorithmic models, which can
% be included in what is called computational intelligence (CI), have
% experienced a recent blossoming thanks to the availability of more and
% more computing means.

% Early applications to air quality forecasting of one of the most
% popular CI methods, i.e., artificial neural networks, were reviewed
% back in 1998 by \cite{Gardner19982627}. Concerning \no2, many other
% applications of CI methods followed, with applications to the air in
% cities of different parts of the world: London
% \cite{gardner_neural_1999}, Santiago de Chile
% \cite{perez_prediction_2001}, Helsinki \cite{kukkonen_extensive_2003},
% Bilbao \cite{agirre-basurko_regression_2006}, Palermo
% \cite{brunelli_two-days_2007}, Athens
% \cite{vlachogianni_evaluation_2011}, ...

% However, the vast majority of the air quality forecasting applications
% and experiments to date deal with what we call ``point-forecasting'',
% as opposed to ``probabilistic forecasting'': they try to predict exact
% future values for different pollutants through modelling their
% conditional mean. There are some obvious disadvantages of this
% approach, which does not consider the inherent uncertainty of the
% predictions and is unsuitable for the cases of heavily skewed data or
% if there is a need to examine certain important strata of the data
% \cite{koenker_quantile_2005}. Quantile regression is a 

% \cite{ziomas_forecasting_1995}

% Probabilistic forecasting. Wind power, stock markets, health...

% Aim of this study.

% Structure of the paper.

% \section{Introduction}
% \label{sec:intro-ii}

Air quality forecasting is an active field of study which grew
steadily during the last three decades and which gathers contributions
from meteorology, physics, chemistry, statistics and computational
intelligence. There are several approaches to air quality forecasting
and they can be roughly divided in two classes: those which rely upon
analyzing the atmosphere status and evolution from a fluid mechanics
and chemical point of view and those which study pollution measures
from a statistical time series analysis and modelling perspective. The
latter can be subsequently divided in approaches that assume that the
data are generated by a given stochastic data model and those that use
algorithmic models and treat the data mechanisms as unknown
\cite{breiman_statistical_2001}. These algorithmic models, which can
be included in what is called computational intelligence (CI), have
experienced a recent blossoming thanks to the availability of more and
more computing means.

However, to date, both the causal modelling and the prediction
research, no matter the approach, have focused mostly on the central
tendencies of the data distribution, i.e. the conditional mean. This
is also the case for CI-based forecasting of air quality
\cite{Gardner19982627,Wang2003555} and its applications to the air in
cities of different parts of the world: London
\cite{gardner_neural_1999}, Santiago de Chile
\cite{perez_prediction_2001}, Helsinki \cite{kukkonen_extensive_2003},
Bilbao \cite{agirre-basurko_regression_2006}, Palermo
\cite{brunelli_two-days_2007}, Athens
\cite{vlachogianni_evaluation_2011}, ...

Although the point forecasts, or forecasts of the central portion of
the conditional distribution, are widely used, there are some obvious
disadvantages associated with them. For example, point forecasting
does not readily inform about the inherent uncertainty of the
predictions, and is generally unsuitable for the cases of heavily
skewed data or if there is a need to examine certain important strata
of the series \cite{koenker_quantile_2005}.

Hence, considering the highly complex nature of the interactions
between meteorological and human factors which affect air quality, it
can be risky to assume that the relationship between those factors and
the concentrations of airborne pollutants is the same for unusually
low concentrations as for unusually high (peak) concentrations. And it
can be even more risky to assume that both relationships are of the
same form as for the central part of the conditional
distribution. Furthermore, there is no need for the explanatory
variables used in forecasting the concentrations of airborne
pollutants on the tails of a conditional distribution to be the same
as the explanatory variables used in forecasting the expected
concentrations or point-forecasts.

This is especially true when forecasting air quality in the framework
of anti-pollution regulations. Most of these regulations indicate that
certain actions must be taken when pollutants exceed thresholds set by
the authorities (as is the case of the European Commission
\cite{european_commission_air_2008}). Modelling the upper quantiles of
the conditional distribution through probabilistic forecasting becomes
a necessity in this case, as we will show. In addition, modelling the
full conditional distribution allows to obtain estimations of the
probability of exceedance of the thresholds, which is a more useful
estimate in terms of communicative power to the general public, as
shown by its extensive use in meteorological forecasting.

In this paper, a first application of probabilistic forecasting to air
quality is presented. The main objective is to demonstrate the
applicability and usefulness of this technique, so future
concentrations of \no in the city of Madrid are forecast in terms of
quantiles and probability of exceedance of certain thresholds. The
rest of the paper is as follows: 


\section{Probabilistic forecasting with quantile regression}
\label{sec:probForec}

As mentioned above, the prediction from most regression models is a
point estimate of the conditional mean of a dependent variable, or
response, given a set of independent variables or predictors. However,
the conditional mean measures only the center of the conditional
distribution of the response, and if we need a more complete summary
of this distribution, for example in order to estimate the associated
uncertainty, quantiles are in order. The 0.5 quantile (i.e., the
median) can serve as a measure of the center, and the 0.9 quantile
marks the value of the response below which reside the 90\% of the
predicted points. Recent advances in computing have inducted the
development of regression models for predicting given quantiles of the
conditional distribution. The technique is called quantile regression
(QR) and was first proposed by Koenker in 1978
\cite{koenker_regression_1978} based on the intuitions of the
astronomer and polymath Rudjer Boscovich in the 18th
century. Elaborating from the same concept of estimating conditional
quantiles from different perspectives, several statistical and CI
models that implement this technique have been developed: from the
original linear proposal to multiple or additive regression, neural
networks, support vector machines, random forests etc.

Quantile regression has gained an increasing attention from very
different scientific disciplines \cite{yu_quantile_2003}, including
financial and economic applications \cite{fitzenberger_economic_2002},
medical applications \cite{soyiri_forecasting_2012}, wind power
forecasting \cite{zhang_review_2014}, electric load forecasting
\cite{7423794,gibbons_quantile_2014}, environmental modelling
\cite{cade_gentle_2003} and meteorological modelling
\cite{bjornar_bremnes_probabilistic_2004} (these references are just
examples and the list is not exhaustive). To our knowledge, despite
its success in other areas, quantile regression has not been applied
in the framework of air quality% , with the exception of
% \cite{martinez-silva_forecasting_2016}
.

As an illustration of the concept (for a detailed discussion of
quantile regression, refer to \cite{koenker_quantile_2005}), given a
set of vectors $(x_i, y_i)$, in point forecasting we are usually
interested in what prediction $\hat y(x) = \alpha_0 + \alpha_1 x$
minimizes the mean squared error,
\begin{equation}
  \label{eq:1}
  E = \frac{1}{n} \sum^n_i \epsilon_i =
  \frac{1}{n} \sum^n_i [ y_i - (\alpha_0 + \alpha_1 x) ]^2.
\end{equation}
This prediction is the conditional sample mean of $y$ given $x$% , that
% is, $\hat y(x) = \hat\alpha_0 + \hat\alpha_1 x$
, or the location of the conditional distribution. But we could be
interested in estimating the conditional median (i.e., the 0.5
quantile) instead of the mean, in which case we should find the
prediction $\hat y(x)$ which minimizes the mean absolute error,
\begin{equation}
  \label{eq:2}
  E = \frac{1}{n} \sum^n_i \epsilon_i =
  \frac{1}{n} \sum^n_i | y_i - (\alpha_0 + \alpha_1 x) |.
\end{equation}
The fact is that, apart from the 0.5 quantile, it is possible to
estimate any other given quantile $\tau$. In that case, instead of
(\ref{eq:2}), we could minimize
\begin{equation}
  \label{eq:3}
E= \frac{1}{n} \sum^n_i f( y_i - (\alpha_0 + \alpha_1 x))
\end{equation}
where
\begin{equation}
  \label{eq:4}
  f(y-q) = \left\{ 
\begin{array}{l l}
\tau (y-q) & \quad \mbox{if $y \ge q$}\\
(1-\tau) (q-y) & \quad \mbox{if $y < q$}\\
\end{array} \right.,
\end{equation}
with $\tau \in (0,1)$. Equation (\ref{eq:3}) represents the
median when $\tau=0.5$ and the $\tau$-th quantile in any other case.

Thus, as we can estimate an arbitrary quantile and forecast its
values, we can also estimate the full conditional distribution, which
will entail us to the results presented in Section \ref{sec:results}.

Among the array of methods that allow to estimate and forecast
data-driven conditional quantiles, in this study we have chosen
quantile regression forests for its ease of use (few parameters have
to be chosen) and for its availability in the free software
mathematical environment \texttt{R} \cite{r-language}.
For a detailed discussion on quantile regression forests, see
\cite{meinshausen_quantile_2006}.

\section{Data description and experimental design}
\label{sec:mm}

\subsection{Protocol for high \no concentration episodes}
\label{sec:madr-prot-high}

Urban air quality is assessed through measuring the concentrations of
several dangerous pollutants including ozone, nitrogen oxides and
particulate matter. While the World Health Organization estimates that
air pollution caused 3.7 million premature deaths worldwide in 2012
\cite{who_ambient_2016}, authorities like the European Union have put
forward regulations to establish limits for the airborne
concentrations of air pollutants \cite{european_commission_air_2008}.

Complying with these regulations, the city of Madrid has a dense
Atmospheric Pollution Monitoring System including 24 stations around
the city. The data gathered by this system are public and are made
available in a hourly basis in a website
\cite{ayuntamiento_de_madrid_sistema}. As mentioned above, recently,
the local government imposed new anti-pollution measures in a protocol
which include traffic restrictions when nitrogen dioxide
concentrations reach the thresholds set by the EU. Concretely, in
Madrid this air quality protocol establishes three action levels that
are raised according to the \no concentrations: a \emph{pre-warning}
for breaches of a threshold of 180 g/m$^3$, a \emph{warning} when
concentrations are over 200 g/m$^3$ and an \emph{alert} when values
over 400 g/m$^3$ are registered. 

% Nitrogen dioxide (\no), together with ozone and particulate matter,
% is a toxic gas which causes significant inflammation of the airways,
% and is the source of nitrate aerosols, which form an important
% fraction of particulate matter and, in the presence of ultraviolet
% light, ozone.
% Epidemiological studies have shown that symptoms of bronchitis in
% children increase in association with exposure to \no
% \cite{pershagen_air_1995,brunekreef_air_1997}. There have been few
% controlled human exposure studies on interactions with other chemical
% pollutants, although studies show that \no exposure enhances
% allergic responses of asthmatics to inhaled pollens
% \cite{strand_nitrogen_1997}.

Given the associated health risks and given the fact that the social
acceptance of the restrictions put in place by the newly established
air quality protocol of Madrid could greatly benefit from proper
prediction of threshold breaches, we will restrict this study to \no
concentrations over the pre-warning threshold. However, it is clear
the results can be readily applicable to other thresholds for \no and
other pollutant species like ozone or particulate matter and their
respective regulatory limits.

\subsection{Nitrogen dioxide data}
\label{sec:no2}

\begin{figure}[tbp]
\begin{center}
 \includegraphics[width=0.8\columnwidth]{img/map.png}
  \caption{Map of Madrid's Atmospheric Pollution Monitoring System
    stations.}
\label{figure:map}
\end{center}
\end{figure}

All of the 24 stations of Madrid's monitoring system, which are
depicted in Figure \ref{figure:map}, capture hourly data for \no. They
are spatially distributed according to European regulations and are
classified into three types: background stations, suburban stations
and traffic stations. The latter measure mainly emissions from
vehicles in their surrounding streets, and they are more prone to
register peak values exceeding the \no safety limits. For this study,
as an example, one of these stations has been selected, namely the
Plaza de Espa\~na station (marked with a letter C in the
figure). 

\begin{figure}[tbp]
\begin{center}
 \includegraphics[width=1.\columnwidth]{img/hourlyConc.eps}
  \caption{Intraday distribution (left) and histogram (right) of the concentration of \no.}
\label{figure:hourlyConc}
\end{center}
\end{figure}

The time series for Plaza de Espa\~na consists of hourly measured
values of the airborne concentrations of \no from 01/01/2000 to
30/11/2015. As can be seen in \ref{figure:hourlyConc}, these values
exhibit a clear intraday pattern, in which the higher values are
located in two peaks around the morning and evening (with highest
average value at 19h) while the nightly hours (from 00h to 05h) have
lower average concentrations. Regarding the peak values over the
threshold pre-warning level of 180 $\mu \mathrm{g/m}^3$, they are
almost never reached during the night. In the same figure we can see
that the distribution of \no presents a positive skew, with values
over the thresholds being rare. In fact, the 99.8\% of the data are
equal or below the pre-warning threshold, while a total amount of 1379
points exceed it.

\subsection{Weather data}
\label{sec:weather-data}

Apart from air quality data, some of the 24 monitoring stations of
Madrid's Atmospheric Pollution Monitoring System also register
meteorological variables, including hourly average temperature and
wind speed, hourly accumulated rainfall and barometric pressure. These
variables encode an approximate picture of the atmospheric situation,
and hence are known to be related to the evolution of the \no
concentrations.

The previously selected Plaza de Espa\~na station registers these
variables except for barometric pressure. However, assuming that the
variations in pressure should not be very significative, we decided,
for this variable, to use data from the closest station that provides
it: Casa de Campo (marked with an M in Figure \ref{figure:map} and
around 3.000 m away from Plaza de Espa\~na). In order to account for
changes in the atmosphere status, the hourly differences of pressure
were used instead of the original series.

\subsection{Experimental design}
\label{sec:experimental-design}

\begin{figure}[tbp]
\begin{center}
 \includegraphics[width=0.6\columnwidth]{img/schema.eps}
  \caption{Summary of the experimental design.}
\label{figure:example}
\end{center}
\end{figure}

Once all the data was gathered, the five series (\no, temperature,
wind speed, rainfall and difference of pressure) were aligned and
merged using their respective time dimension, and UTC times were used
for all of them. Then, each of them was lagged in order to create new
variables in the following manner: for \no, a set of embedding
dimensions was selected with the objective to have into account the
recent past hours, the analogous hours of the day before and the
analogous hours of the same day of the week before, that is,
$\mathbf{d}_{\mathrm{NO}_2} = (1,2,3,4,24,25,26,27,168,169,170,171)$.
For the meteorological variables, we selected only the past hours:
$\mathbf{d}_{\mathrm{met}}= (1,2,3,4)$. Then, for each hour $t$ in the
15 years covered by the data, a vector was constructed as
$\mathrm{\mathbf{z}}_t = (y_{t-\mathbf{d}_{\mathrm{NO}_2}},
x^{\mathrm{Temp}}_{t-\mathbf{d}_{\mathrm{met}}},
x^{\mathrm{Wind}}_{t-\mathbf{d}_{\mathrm{met}}},
x^{\mathrm{Rain}}_{t-\mathbf{d}_{\mathrm{met}}},
x^{\mathrm{Press}}_{t-\mathbf{d}_{\mathrm{met}}}; y_t)$.

The time of day (as proven by Figure \ref{figure:hourlyConc}), the
weekday (as is known that during the weekends there are less vehicles)
and the day of the year (as seasonal effects are also present) were
also included as dummy variables in the set of predictors. As a
result, a matrix with 139.513 rows and 33 columns (32 predictor
variables and one response variable $y_t$) was constructed.

% However, as can be also seen in Figure \ref{figure:hourlyConc}%
% % (page \pageref{figure:hourlyConc})
% , during the nightly hours almost no \no points exceeding the
% thresholds were registered. Given the fact that we are mostly
% interested in studying the values around these
% thresholds, and considering the huge volume of data available%  (with
% % 139.513 points just in the \no series)
% , one third of the data,
% corresponding to the hours from 00h to 07h of the response variable,
% were removed from the final series. However, the rest of the
% variables kept lagged values for these hours, accounting for
% possible night effects on the peaks.
% In the same spirit, we removed from the data those months in which the
% \no values never reached the first threshold of 180 g/m$^3$.

Reserving a part of the data to check that the models are able to
generalize is a usual procedure when dealing with statistics or
computational intelligence models. In order to have enough peak data
to allow inference, we divided our data in two blocks: a block
from 01/01/2000 to 31/12/2009 will be used to train the
models, and the rest of the data, from 01/01/2010 to 31/12/2015 will
be used to test their properties with unseen data.

\subsection{Evaluation of probabilistic forecasts}
\label{sec:eval-prob-forec}

The evaluation of probabilistic forecasts is a complex problem and an
open issue in the literature \cite{mcsharry_methodology_2009}. On the
one hand, given the fact that we can derive expected values from
probabilistic forecasts, we can use the standard metrics for
evaluating point forecasts, as root mean squared error (RMSE), mean
average error (MAE), correlation or bias.

These measures can be compared against some reference models, to
establish how the probabilistic forecasts perform when forecasting
expected values in the central part of the distribution. In this work,
we used three point forecasts models as benchmark: persistence, a
linear regression and random forests.

Persistence, also known as naive predictor, is the
simplest way of producing a forecast. The persistence forecasts are
obtained by issuing the last observation as the forecast for all
future horizons: $\hat y_{t+k|t} = y_t$. This method works well when
the patterns of the series change very slowly with time.

Another simple method of producing forecasts is through a least-squares linear
regression which combines the predictors $\mathbf{z}_t$ (as defined in
section \ref{sec:experimental-design}) in a linear manner to compute
estimations for future values of the series.

Finally, to obtain a more state-of-the-art benchmark, a nonlinear
model from the machine learning field was selected and trained to
point-forecast future values of the series. Among the panoply of
nonlinear machine learning models available, we selected Random
Forests \cite{breiman_random_2001}. We used the \texttt{R}
implementation \cite{randomforests} with its default parameters.

However, on the other hand, if non-central regions of the forecast
distributions are to be evaluated, more elaborated metrics must be
used.
%
% \subsubsection{Reliability and sharpness}
% \label{sec:reli-sharpn}
%
Amongst the main attributes of probabilistic forecasts used for
evaluation, reliability and sharpness are two of the most common.

Reliability deals with how close the actual distribution of the data
is to the predicted one and is related to the unconditional coverage
of a prediction interval. That is, if a predicted 50\% interval covers
50\% of the observed load values, then it is considered reliable. In a
way, reliability for probabilistic forecasting is similar to bias for
point forecasting, a high reliability corresponding to a low bias.

However, reliability is not sufficient to characterize the quality of
a probabilistic forecast since a forecast based on climatology is
perfectly reliable and yet has no skill. A model is said to have no
skill when it provides the same forecast distribution for all
situations. A skillful model will provide sharper distributions for
more certain situations and wider distribution when the uncertainty on
the outcome is higher. 

Sharpness deals with how tightly the predicted distribution covers the
actual one. A 95\% interval is said to be sharp if the maximum and
minimum values of the observed values are very close to the upper and
lower bounds of the predicted interval. Sharpness for probabilistic
forecasts is similar to the variance of errors for point forecasting,
where a high sharpness corresponds to a low variance of
errors.

Reliability is related to sharpness in the same way bias is
related to variance in deterministic forecast evaluation. That is,
there is usually a sharpness-reliability performance trade-off in the
same way that there is a bias-variance trade-off for point forecast
models.

\subsection{Evaluation of alert forecasting}
\label{sec:eval-extr-value}

The evaluation of binary forecasts (forecasts of events which might
occur or not) is usually performed through summary statistics of the
contingency table, from which different ways of measuring the goodness
of an alert forecast have been proposed
\cite{mcsharry_methodology_2009}.  A contingency table arranges the
four different outcomes which are expected from an alert forecast:
true positives (the alert is forecast and it actually happens, noted
as $TP$), false positives (the alert is forecast but it does not
happen, $FP$), true negatives (no alert is forecast and none
happens, $TN$) and false negatives (no alert is forecast but
it actually happens, $FN$).

Two of the most popular performance measures are specificity and
sensitivity. Specificity, also known as true negative rate, measures
the proportion of negatives that are correctly forecast as such:
$\mathrm{\emph{spec}} = TN / (TN + FP)$.  Sensitivity, also called true
positive rate, is the fraction of positives that were correctly
forecast: $\mathrm{\emph{sens}} = TP / (TP + FN)$.

However, predicting extreme values is a special situation in which the
imbalance of the different types of events plays a crucial role
\cite{ferro_deterministic_2011}.  Both specificity and sensitivity
behave poorly in the framework of problems which have few true
positives with respect to the total, which is our case. Hence, some
other measures have been proposed to account for imbalanced
contingency tables:
\begin{description}
\item[Balanced accuracy] Proposed by \cite{brodersen2010balanced}, it
  can be defined as the average accuracy (number of correct
  predictions divided by number of predictions) obtained in each
  class:
  $\mathrm{\emph{BAcc}} = \frac{1}{2} \left(\frac{TP}{TP+FN} +
    \frac{TN}{TN+FP}\right)$. It ranges from 0 to 1, with 0 indicating
  no accuracy.
\item[True skill statistic] Also known as Hanssen-Kuiper skill
  statistic, measures how well did the forecast separate the positive
  from the negative events:
  $\mathrm{\emph{TSS}} = \frac{TP}{TP + FN} - \frac{FP}{TN+FP}$. It ranges from -1 to
  1, and 0 indicates no skill.
\item[Extreme dependency score] Compares the fraction of the observed
  events with the fraction of the correctly forecast events. It ranges
  from -1 to 1, perfect score is 1. It
  does not tend to zero for rare events \cite{stephenson2008extreme}:
  \begin{equation}
    \label{eq:6}
    \mathrm{\emph{EDS}} = 2 \frac{\ln(\frac{TP+FP}{n})}{\ln(\frac{TP}{n})} - 1,
  \end{equation}
  where $n= TP+FN+TN+FN$.
\end{description}

Finally, another interesting tool to evaluate alert forecasts is the
relative operating characteristic or ROC curve, which is a graph that
illustrates the performance of a binary classifier system as its
discrimination threshold is varied. The curve is created by plotting
sensitivity against ($1 -$ specificity).


\section{Results and discussion}
\label{sec:results}

We performed four different experiments, and their results are shown
in this section. First we will compare the expected values obtained
through probabilistic forecasts with reference point forecasts
models. Secondly, we will study the relative importance associated
with the variables used in the first experiment, with the aim to
discover any distinctive pattern in how each variable is used for the
different regions of the forecast distributions. In the third
experiment, we center our attention in the prediction of the extreme
values which exceed the regulation's thresholds, while in the fourth
we show how the probability of alerts can be predicted and presented
in more informative ways.

\subsection{Reference models}
\label{sec:deterministic}

In the first experiment, we used quantile regression to compute
point-forecasts of the expected value (median) for one-day ahead
predictions of \no concentrations.

% latex table generated in R 3.2.2 by xtable 1.7-4 package
% Thu Mar 31 18:48:22 2016
\begin{table}[tbp]
\caption{\label{tab:determ}Point forecast error measures for reference
models (persistence, linear regression, random forests and median of
the probabilistic model (QRF).}
  \centering
\begin{tabular}{rrrrr}
  \toprule
 & RMSE & MAE & Bias & Corr \\ 
  \midrule
  Persistence & 13.47 & 9.23 & 0.04 & 0.88 \\ 
  LR   & 11.51 & 8.16 & -1.62 & 0.91 \\ 
  RF   & 11.27 & 7.89 & -2.14 & 0.92 \\
  Q50  & 11.30 & 7.63 & -0.27 & 0.91 \\ 
   \bottomrule
\end{tabular}
\end{table}

Table \ref{tab:determ} shows the values of the root mean squared error
(RMSE), mean average error (MAE), bias and correlation for the
aforementioned reference models and the median forecast by the
probabilistic model. As we can see, the median-based model Q50 behaves
well in general compared to the other models, being especially good in
terms of MAE and bias. This might be related to the median being more
robust than the mean in the presence of outliers.

However, in this framework, we are, as a matter of fact, interested in
those outliers, as they precisely are the values which trigger the
activation of the air quality protocol.

\subsection{Variable importance}
\label{sec:vble}

\begin{figure}[tbp]
\begin{center}
 \includegraphics[width=.85\columnwidth]{img/varImpPlot.png}
 \caption{Importance of the variables in quantiles 0.05, 0.50 and
   0.95, in descending order according to the importance in the higher
   quantile.
   % The dotted line marks the limit for variables with importance
   % equal to 2 in quantile 0.95.
 }
\label{figure:varImpPlot}
\end{center}
\end{figure}

Quantile regression forests offer a straightforward method for feature
selection: the method allows for the direct calculation of the mean
decrease in accuracy, which is a measure of the variable
importance. We will use it in this second experiment to investigate
which variables are more important for the problem under
consideration.

The idea is to directly measure the impact of each feature on the
accuracy of the model, and this is done through a random alteration of
the values of each feature. Once this alteration is done, it is
possible to measure how much it decreases the accuracy of the
model. For unimportant variables, this should have little or no effect
on the model accuracy, while altering important variables should
significantly decrease it.

In Figure \ref{figure:varImpPlot} we can see the importance of each of
the 31 variables considered by the model according to their mean
decrease in accuracy, for quantiles 0.05, 0.5 and 0.95. The variables
are sorted according to their importance in the latter quantile, as
this is the region of the distribution that we are most interested
in. To overcome the high computational costs of the method to obtain
variable importance and to account for the randomness which is
inherent to the model, the importance values were computed for 30
random samples of 100 vectors $\mathrm{\mathbf{z}_t}$, as defined in
Section \ref{sec:experimental-design}. In the figure, the average
importance for each variable over the 30 runs is shown.

Unsurprisingly, the \no value measured the hour before (noted
\textsf{NO2\_1} in the graph) is clearly the most important variable
in the three quantiles. This is closely related to the principle
behind the persistence or naive predictor, and hence suggests that the
improvements over this method are to be obtained by properly
exploiting the rest of the variables which complement the predictive
power of the value of \no measured at $t-1$. The value of \no measured
at the same hour of the day before, \textsf{NO2\_24} in the graph, is
the second variable in importance for all the three considered
quantiles, which accounts for the intra-daily effect shown in Figure
\ref{figure:hourlyConc}.

If we center our attention in the 0.95 quantile, it is remarkable that
the difference of barometric pressure (\textsf{difPres}) does not seem
to be as important as expected, although it is clearly more important
in the upper quantile than in the other two. This might be considered
counter-intuitive, as the atmospheric stability is one of the
circumstances known to be usually required for high \no concentrations
in the air. Concerning wind, we can see how the wind speed registered
at time $t-1$ (\textsf{WindSpeed\_1}) is of high importance while
other, more old wind speed measures are not. This hints at the
immediate washing effect of high wind speeds, which carry \no outside
the city (its effect being even more important for quantile
0.05). Finally, we can observe that \textsf{Rainfall} can be
considered unimportant for \no forecasting.

Finally, it is interesting to highlight the differences amongst the
three considered quantiles, which prove the need and usefulness of
quantile forecasting of \no. The distribution of variable importances
for the median quantile, in particular, shows a separate pattern from
the upper and lower quantiles, being the case that the difference
between the value of \no at time $t-1$ and the rest of the variables
is much more acute for the median quantile than for the others.

\subsection{Probabilistic forecasting of extreme values}
\label{sec:probabilistic}

\begin{figure}[tbp]
\begin{center}
  \includegraphics[width=1.\columnwidth]{img/reliability-sharpness.png}
  \caption{\label{fig:reliab}Reliability (left) and sharpness (right).}
\end{center}
\end{figure}

The third experiment consisted in predicting the full distribution of
\no concentrations. 

In the left part of Figure \ref{fig:reliab}, we see that the
reliability of the forecasts is satisfactory for all considered
probabilities. The model predictions are centered around the ideal
values, although the model slightly over-predicts for low
probabilities and under-predicts for higher probabilities. The slight
under-prediction for high quantiles is not necessarily negative in the
frame of \no forecasting: it means that the model is slightly
conservative which can be considered a good property since the
thresholds should not to be exceeded and the cost of false alarms is
high.

In the right part of Figure \ref{fig:reliab}, the sharpness of the
forecasts, for five different coverage rates (90\%, 70\%, 50\%, 30\%
and 10\%) is presented. As expected, the median inter-quantile range
decreases with the coverage rate. For all coverage rates the
inter-quantile range increases with forecast horizon due to the
increasing forecast uncertainty. Also, the forecast can be said to be
skillful since the minimum and especially the maximum observed
inter-quantile distances are significantly different from the median
values.

\begin{figure*}[tbp]
\begin{center}
  \includegraphics[width=1.\textwidth]{img/q95forecast.png}
  \caption{Forecasts for quantile 0.95. The horizontal line marks the
    threshold of 180, while the grey line represents the
    forecast values for quantile 0.95, small dots are observed values
    and triangles represent correctly forecast peak values.}
  \label{figure:q95forec}
\end{center}
\end{figure*}

In order to show the performance of the model when forecasting extreme
values, Figure \ref{figure:q95forec} shows the 0.95 quantile forecasts
for the last six years of the \no series. As we can see, the forecasts
for this model (represented by the gray line) acts as an upper
envelope for the series, which results in peaks over the 180 g/m$^3$
warning threshold being properly forecast in many cases. This is
especially true in the event of high concentrations episodes, as those
occurred in late 2014 and around the start of 2015, for example. 

\begin{table}[tbp]
  \centering
  \caption{\label{tab:results}Performance measures (as defined in
    Section \ref{sec:eval-extr-value}) for the considered
    models over the test set.}
\begin{tabular}{rrrrrr}
  \toprule
 & persist. & LR & Q50 & Q95 & th=180 \\%& th=200 \\ 
  \midrule
  % total & 47997 & 47997 & 47997 \\%& 47997 \\ 
  \emph{TP} & 24 & 25 & 15 & 59 & 64 \\%& 62 \\ 
  \emph{FP} & 45 & 27 & 12 & 309 & 1006 \\%& 1030 \\ 
  \emph{TN} & 47.884 & 47.902 & 47.917 & 47.620 & 46.923 \\%& 46899 \\ 
  \emph{FN} & 44 & 43 & 53 & 9 & 4 \\%& 6 \\
  \midrule
  \emph{spec} & 1.00 & 1.00 & 1.00 & 0.99 & 0.98 \\%& 0.98 \\ 
  \emph{sens} & 0.35 & 0.37 & 0.22 & 0.87 & 0.94 \\%& 0.91 \\ 
  % accuracy & 1.00 & 0.99 & 0.98 & 0.98 \\ 
  \emph{BAcc} & 0.68 & 0.68 & 0.61 & 0.93 & 0.96 \\%& 0.95 \\ 
  % positive predictive value & 0.56 & 0.16 & 0.06 & 0.06 \\ 
  % frequency bias & 0.40 & 5.41 & 15.74 & 16.06 \\ 
  % false alarm ratio & 0.44 & 0.84 & 0.94 & 0.94 \\ 
%  false positive rate & 0.00 & 0.00 & 0.00 & 0.01 & 0.02 \\%& 0.02 \\ 
%  true positive rate & 0.35 & 0.37 & 0.22 & 0.87 & 0.94 \\%& 0.91 \\ 
  \emph{TSS} & 0.35 & 0.37 & 0.22 & 0.86 & 0.92 \\%& 0.89 \\ 
  \emph{EDS} & 0.73 & 0.74 & 0.63 & 0.96 & 0.98 \\%& 0.97 \\ 
  % f-score & 0.20 & 0.27 & 0.11 & 0.11 \\ 
   \bottomrule
\end{tabular}
\end{table} 

Table \ref{tab:results} shows the contingency table and the
performance measures described in Section \ref{sec:eval-extr-value}
for the various models considered. The first two columns correspond to
the persistence and the linear regression models. The third and fourth
column correspond to the 0.50 and 0.95 quantiles, respectively. The
fifth column will be analized below.

The first thing we verify in the table is that, for all the models,
the amount of true negatives (values below the threshold) largely
exceeds the amount of true positives. This causes the specificity to
be trivially almost perfect in all the cases, an evidence of why this
measure is not very useful in this context.

However, in terms of balanced accuracy (\emph{BAcc}), true skill score
(\emph{TSS}) and extreme dependency score (\emph{EDS}), there is a
clear pattern which differentiates the different mean or median
forecasts from the 0.95 quantile forecast. As expected, the latter
manages to predict a higher number of true positives, whereas
incurring in far less false negatives. This is reflected by the three
performance measures, the 0.95 quantile forecast consistently
outperforming the other models.

As stated above, quantile regression allows for the prediction of any
arbitrary quantile. Hence, in order to predict the protocol alerts,
corresponding to the 180 $\mathrm{\mu g/m}^3$ threshold, we might as
well compute which quantile does this threshold define in the full
distribution of the data. In fact, it corresponds to the 0.995
quantile, and although conventional large sample theory for quantile
regression does not apply sufficiently far in the tails (for such
extreme quantiles there usually are not enough data to guarantee that
the model is correct), we have used QRF to predict it: it corresponds
to the fifth column of Table \ref{tab:results}. It is shown in this
column that the model manages to predict extreme values outperforming
the 0.95 quantile in all the measures except for specificity. Out of
the 68 alerts registered, this model manages to properly predict 64,
although it also produces around three times more false positives than
the 0.95 quantile.

\begin{figure}[tbp]
\begin{center}
 \includegraphics[width=1\columnwidth]{img/ROC.png}
 \caption{ROC curve for the two considered quantile forecasts. The
   dashed lines indicate the location of the (sensitivity,
   1-specificity) point corresponding to a threshold of 180
   $\mathrm{\mu g/m}^3$.}
\label{figure:ROC}
\end{center}
\end{figure}

Finally, in \ref{figure:ROC}, the ROC curves for the 0.50 and 0.95
quantile forecasts are shown. As we can see, the point of the curve
corresponding to the threshold of 180 $\mathrm{\mu g/m}^3$ (marked by
the dashed lines) shows much better properties in the case of the 0.95
quantile than for the 0.50 quantile.

\subsection{Forecasting the probability of alerts}
\label{sec:alertProb2}

\begin{figure}[tbp]
\begin{center}
  \includegraphics[width=0.95\columnwidth]{img/ep_nov2015.png}
  \caption{For a period in November, 2015, the upper graph represents
    observed values (red line), alert threshold (dashed horizontal
    line) and distribution of the forecasts (boxplots). The lower
    graph shows probability of exceedance of the threshold.}
\label{figure:PoE}
\end{center}
\end{figure}

Following \cite{raftery_use_2014}, there are different types of users
and applications of probabilistic forecasts, and this suggests that it
is important for developers of probabilistic forecasts to interact
with the users to agree on the presentation of forecasts which is more
useful to them.

The fourth experiment is an example of an elaboration over the
probabilistic results of the models that could help decision-making
authorities to better understand the forecasts and adapt their
policies to them. Figure \ref{figure:PoE} shows, in two different
fashions, the predictions for a period in November, 2015, in which
a high \no concentrations episode was registered.

The upper part of the figure is what is known in the meteorology field
as an ``EPSgram''. EPSgrams (which take its name from the Ensemble
Prediction System of the ECMWF) display the time evolution of the
distribution of a magnitude. In the graph, we can see how the forecast
\no distribution varied during the episode.  The forecast distribution
at each forecast time is represented by a box-and whiskers plot
showing the median (short horizontal line), the 25th and 75th
percentiles (vertical box) and 5th and 95th percentiles (vertical
lines).

Obviously, although it conveys a large and comprehensive amount of
information, the representation of the forecasts through EPSgrams
requires a scientific knowledge that might not be available to
decision-makers nor citizens. In order to provide them with a more
easily interpretable and understandable representation, the lower
part of Figure \ref{figure:PoE} shows, for each forecast, the
probability of exceedance for the 180 $\mathrm{\mu g/m}^3$ threshold
which triggers the first alert in the \no protocol.

This predicted probability of exceedance, which can be represented
through several graphical solutions (for example as a pie chart or
equivalent) is proven to be a useful and understandable way to publish
this type of information. Furthermore, meteorological services make
use of it in the dissemination of the weather forecasts, which in turn
implies that the public is already familiarized with it.

\section{Conclusions}
\label{sec:concl}

In this paper, a first application of probabilistic forecasting to the
problem of predicting extreme \no pollution episodes has been
presented. Data from the air quality sensor network of the city of
Madrid have been used to develop quantile regression models tailored
to predict one hour-ahead \no concentrations in a urban location.

Through four different experiments, it has been shown that
probabilistic forecasting using quantile regression has advantages
over traditional point-forecasting, allowing for a better prediction
of extreme concentrations and also a more insightful representation of
the predictions.

In addition to the median, the proposed approach allows for the
prediction of the whole distribution of the future values of \no
concentrations. This is especially useful when, as is the case, the
tails of the distribution are of interest, allowing for a more precise
prediction of extreme values.

We have shown how, when used to predict the median, the proposed model
compares favourably to point-forecasting approaches including simple
models as persistence or linear regression as well as complex and
highly nonlinear models as random forests.

On the other hand, we have shown the accuracy and usefulness of the
probabilistic predictions, which have good sharpness and
reliability. We have used the forecast upper quantiles to predict
high \no concentrations bond to produce alerts in the air quality
protocol of the municipality of Madrid, and the results indicate that
our proposal is able to predict these episodes in a much more
dependable way.

As a by-product of the chosen model, we performed a study about the
relative importance of the (autoregressive and meteorologic)
independent variables available, confirming, in a purely data-driven
approach, some of the assumptions made by other authors about the
atmospheric interactions affecting the \no concentrations.

Finally, we have shown how probabilistic forecasts can be represented
in an comprehensible and intelligible way, thus allowing authorities
and decision-makers, as well as the general public, to make a more
beneficial use of the predictions.

Extensions to this work, which has been considered by the Municipality
of Madrid to renew its predictive operational models, are under
development, including spatio-temporal considerations, longer
forecasting horizons and the inclusion of other covariates including
numerical predictions.

\section*{References}

\bibliography{refs_nourl}

\end{document} 
%%% Local Variables:
%%% mode: latex
%%% TeX-master: t
%%% End:
